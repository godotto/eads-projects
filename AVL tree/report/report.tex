\documentclass{article}

\usepackage{listings}
\usepackage{xcolor}
\usepackage{geometry}

\title{AVL tree based dictionary - an EADS project}
\author{Maciej Marcinkiewicz}
\date{January 2020}

\newgeometry{lmargin=3.2cm, rmargin=3.2cm, bmargin=2.5cm}

\lstset{
    tabsize=4, % tab space width
    showstringspaces=false, % don't mark spaces in strings
    % numbers=left, % display line numbers on the left
    basicstyle=\ttfamily,
    keywordstyle=\color{blue}\ttfamily,
    stringstyle=\color{red}\ttfamily,
    commentstyle=\color{green}\ttfamily,
    % morecomment=[l][\color{magenta}]{\#}
}

\begin{document}

\maketitle

\section{Introduction}

The aim of the project is to implement an AVL tree based dictionary in the C++ programming
language with use of template feature. As it is dictionary, the main class has two 
generic parameters – {\tt Key} and {\tt Info}. We assume that {\tt Key} is unique
and if new data with already existing key is being aded, {\tt Info} is replaced with new
value. Additional requirement is to print the structure in a way that root is on the left
hand side of screen and leaves are on the right hand side.

\section{Overview of funtions}

\subsection{Constructors and destructor}

\begin{lstlisting}[language=C++]
Tree()
\end{lstlisting}
Default constructor.

\begin{lstlisting}[language=C++]
~Tree()
\end{lstlisting}
Destructor.

\begin{lstlisting}[language=C++]
Tree(const Tree<Key, Info> &other)
\end{lstlisting}
Copy constructor.

\begin{lstlisting}[language=C++]
Node(const Key &key, const Info &info)
\end{lstlisting}
Constructor for {\tt Node}. Creates new {\tt Node} object with given key and information.

\begin{lstlisting}[language=C++]
~Node()
\end{lstlisting}
Destructor for {\tt Node}.

\vspace{\baselineskip}

\subsection{Overloaded operators}

\begin{lstlisting}[language=C++]
Tree<Key, Info> &operator=(const Tree<Key, Info> &other)
\end{lstlisting}
Assignment operator.

\begin{lstlisting}[language=C++]
bool operator==(const Tree<Key, Info> &other)
\end{lstlisting}
Comparison operator. Returns {\tt true} if compared trees contain the same elements.

\begin{lstlisting}[language=C++]
bool operator!=(const Tree<Key, Info> &other)
\end{lstlisting}
Comparison operator. Returns {\tt false} if compared trees contain the same elements.
\newpage

\subsection{Operations on tree}

\begin{lstlisting}[language=C++]
int Height(Node *node) const
\end{lstlisting}
Returns height of {\tt Node}. If {\tt Node} is equal to {\tt nullptr} returns -1.

\begin{lstlisting}[language=C++]
int GetBalance(Node *node) const
\end{lstlisting}
Returns balance factor of {\tt Node}. If {\tt Node} is equal to {\tt nullptr} returns -1.

\begin{lstlisting}[language=C++]
int UpdateHeight(int leftHeight, int rightHeight) const
\end{lstlisting}
Returns updated high for {\tt Node}. {\tt leftHeight} and {\tt rightHeight} are heights
of node's children.

\begin{lstlisting}[language=C++]
Node *SingleRotateLeft(Node *node)
Node *SingleRotateRight(Node *node)
Node *DoubleRotateLeft(Node *node)
Node *DoubleRotateRight(Node *node)
\end{lstlisting}
These methods perform operations of nodes rotation.

\begin{lstlisting}[language=C++]
Node *GetMinimalKey(Node *node) const
\end{lstlisting}
Returns {\tt Node} with smallest key value.
\vspace{\baselineskip}

\subsection{Modifiers}

\begin{lstlisting}[language=C++]
public: void Insert(const Key &key, const Info &info)
private: Node *Insert(Node *parent, 
                      const Key &key, const Info &info)
\end{lstlisting}
Add new element with given data. If {\tt key} exists in tree, method replaces {\tt info}.

\begin{lstlisting}[language=C++]
public: void Remove(const Key &key);
private: Node *Remove(Node *node, const Key &key)
\end{lstlisting}
If {\tt key} exists in tree, method removes the entry with this key.

\begin{lstlisting}[language=C++]
public: bool Clear()
private: void Clear(Node *node)
\end{lstlisting}
Removes all nodes. If tree was empty before, method returns {\tt false}.
\vspace{\baselineskip}

\subsection{Printing}

\begin{lstlisting}[language=C++]
public: void PrintInorder() const
private: void PrintInorder(Node *node) const
\end{lstlisting}
Prints tree within in-order traversal.

\begin{lstlisting}[language=C++]
public: void PrintPreorder() const
private: void PrintPreorder(Node *node) const
\end{lstlisting}
Prints tree within pre-order traversal.

\begin{lstlisting}[language=C++]
public: void PrintPostorder() const
private: void PrintPostorder(Node *node) const
\end{lstlisting}
Prints tree within post-order traversal.
\newpage

\begin{lstlisting}[language=C++]
public: void PrintVisually() const
private: void PrintVisually(Node *node) const
\end{lstlisting}
Prints tree's visual representation.

\subsubsection*{Example}
\begin{lstlisting}[language=C++]

                                (11, 1)
                                /
                         (10, 1)
                        /
                 (9, 1)
                       \
                 /      (7, 1)
                /
root---> (3, 1)
                \
                 (-1, 1)
                         \
                         (-12, 1)
\end{lstlisting}
\vspace{\baselineskip}

\subsection{Other methods}

\begin{lstlisting}[language=C++]
bool IsEmpty() const
\end{lstlisting}
Returns {\tt true} if tree contains no nodes.

\begin{lstlisting}[language=C++]
int Size() const
\end{lstlisting}
Returns private field {\tt size} which contains number of nodes in tree.

\begin{lstlisting}[language=C++]
bool Search(const Key &key) const
Node *Search(Node *node, const Key &key) const
\end{lstlisting}
Returns {\tt true} if {\tt key} is in the tree.

\begin{lstlisting}[language=C++]
void CopyTree(Node *&newTree, Node *copiedTree)
\end{lstlisting}
Auxiliary method used by copy constructor and assignment operator. Copies tree to given
root {\tt newTree}.

\begin{lstlisting}[language=C++]
void CompareTrees(Node *left, Node *right, bool &isEqual)
\end{lstlisting}
Auxiliary method used by comparison operators. If trees are not the same saves {\tt false}
value to {\tt isEqual} variable.
\vspace{\baselineskip}

\section{Decisions and changes}

My conception did not have any bigger changes. On the beginning I wanted to implement
methods only for single rotations. If double rotations were needed, I would perform them
within {\tt Insert} and {\tt Remove}. However adding double rotation methods increased
readability of code.


\end{document}