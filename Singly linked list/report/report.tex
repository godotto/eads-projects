\documentclass{article}

\usepackage{indentfirst}
\usepackage{listings}
\usepackage{xcolor}
\usepackage{geometry}

\title{Singly linked list - an EADS project}
\author{Maciej Marcinkiewicz}
\date{November 2019}

\newgeometry{lmargin=3.2cm, rmargin=3.2cm, bmargin=2.5cm}

\lstset{
    tabsize=4, % tab space width
    showstringspaces=false, % don't mark spaces in strings
    % numbers=left, % display line numbers on the left
    basicstyle=\ttfamily,
    keywordstyle=\color{blue}\ttfamily,
    stringstyle=\color{red}\ttfamily,
    commentstyle=\color{green}\ttfamily,
    % morecomment=[l][\color{magenta}]{\#}
}

\begin{document}

\maketitle

\section{Introduction}

The aim of the project was to implement singly linked list in the C++ programming
language with use of class template mechanism. Template was meant to accept two generic
variables - {\tt Key} and {\tt Info} which are the fields of each list's node. The first
one's purpose is to behave as identifier, i.e. operations such as removing or inserting
nodes are based on these. The latter contains information that is meant to be stored
in nodes.

\indent
The second task of project was to implement funtion {\tt Merge} which combines two lists
in specific way and returns the result of that operation. The restrictions were to not
include the funtion as a member of {\tt Sequence} class (which is the name of list class)
and to not use a keyword {\tt friend}.

\section{Overview of funtions}

\subsection{Constructors and destructor}

\begin{lstlisting}[language=C++]
Sequence()
\end{lstlisting}
Default constructor.

\begin{lstlisting}[language=C++]
~Sequence()
\end{lstlisting}
Destructor.

\begin{lstlisting}[language=C++]
Sequence(const Sequence<Key, Info> &other)
\end{lstlisting}
Copy constructor.

\begin{lstlisting}[language=C++]
Node(const Key &key, const Info &info)
\end{lstlisting}
Constructor for {\tt Node}. Creates new node with given key and information.
\vspace{\baselineskip}

\subsection{Overloaded operators}

\begin{lstlisting}[language=C++]
Sequence<Key, Info> &operator=(const Sequence<Key, Info> &other)
\end{lstlisting}
Assignment operator.

\begin{lstlisting}[language=C++]
bool operator==(const Sequence<Key, Info> &other)
\end{lstlisting}
Comparison operator. Returns {\tt true} if compared lists contain the same elements.

\begin{lstlisting}[language=C++]
bool operator!=(const Sequence<Key, Info> &other)
\end{lstlisting}
Comparison operator. Returns {\tt false} if compared lists contain the same elements.
\newpage

\begin{lstlisting}[language=C++]
Sequence<Key, Info> operator+(const Sequence<Key, Info> &other)
\end{lstlisting}
Returns combination of two list.

\begin{lstlisting}[language=C++]
Sequence<Key, Info> &operator+=(const Sequence<Key, Info> &other)
\end{lstlisting}
Extends list by another list's elements.

\subsection{Adding elements}

\begin{lstlisting}[language=C++]
void PushFront(const Key &key, const Info &info)
\end{lstlisting}
Method which adds new node of given key and information on the beginning of list.

\begin{lstlisting}[language=C++]
void PushBack(const Key &key, const Info &info)
\end{lstlisting}
Method which adds new node of given key and information on the end of list.

\begin{lstlisting}[language=C++]
bool Insert(const Key &key, const Info &info, const Key &after,
            int whichOccurance = 1)
\end{lstlisting}
Method which adds new node of given key and information after node with specified key.\\
{\tt whichOccurance} tells after which occurance of key {\tt after} new node should
be inserted. Returns {\tt true} if operation was successful. Returns {\tt false} if
the proper node to insert after does not exist.
\vspace{\baselineskip}

\subsection{Removing elements}

\begin{lstlisting}[language=C++]
bool PopFront()
\end{lstlisting}
Method which removes the first node. Returns {\tt false} if list is empty.

\begin{lstlisting}[language=C++]
bool PopBack()
\end{lstlisting}
Method which removes the last node. Returns {\tt false} if list is empty.

\begin{lstlisting}[language=C++]
bool Remove(const Key &key, int whichOccurance = 1)
\end{lstlisting}
Method which removes node of given key. {\tt whichOccurance} tells which occurance
of key in list should be removed. Returns {\tt false} if list is empty or there is
no proper node to remove.

\begin{lstlisting}[language=C++]
bool Clear()
\end{lstlisting}
Removes all nodes. Returns {\tt false} if list is empty.

\begin{lstlisting}[language=C++]
bool RemoveAllOccurances(const Key &key)
\end{lstlisting}
Method which removes all nodes with {\tt key}. Returns {\tt false} if list is empty
or there are no nodes with such a key.
\vspace{\baselineskip}

\subsection{Other methods}

\begin{lstlisting}[language=C++]
bool IsEmpty() const
\end{lstlisting}
Returns {\tt true} if list contains no nodes.

\begin{lstlisting}[language=C++]
int Size() const
\end{lstlisting}
Returns private field {\tt size} which conatains number of nodes in list.

\begin{lstlisting}[language=C++]
bool Find(const Key &key, int whichOccurance = 1) const
\end{lstlisting}
Returns {\tt true} if there is node of given key in the list. {\tt whichOccurance}
tells which occurance of key has to be found.

\begin{lstlisting}[language=C++]
int NumberOfOccurances(const Key &key) const
\end{lstlisting}
Returns number of occurances of nodes with given key.

\begin{lstlisting}[language=C++]
Sequence<Key, Info> GetFragmentOfSequence(int start,
                                          int length) const
\end{lstlisting}
Returns a list which is a part from the list in which this method has been called.
{\tt start} is a position of the first element taken and {\tt length} indicates
how many elements should be taken.

\begin{lstlisting}[language=C++]
void Print() const
\end{lstlisting}
If possible, prints all elements. If list is empty prints message which informs about that.

\begin{lstlisting}[language=C++]
void Print(std::ostream &os) const
\end{lstlisting}
The same as above with an addition of ostream as an argument. Message about how successful
operation of printing was is passed into this ostream to make unit testing easier.

\vspace{\baselineskip}

\subsection{Merge function}

\begin{lstlisting}[language=C++]
Sequence<Key, Info> Merge(const Sequence<Key, Info> &sequence1,
                          int start1, int len1,
                          const Sequence<Key, Info> &sequence2,
                          int start2, int len2,
                          int count)
\end{lstlisting}
Returns {\tt Sequence} which is a product of two other lists. {\tt start1} and
{\tt start2} tell how many nodes from the beginning of lists should be skipped,
{\tt len1} and {\tt len2} tell how many nodes should be taken from each of lists
and finally {\tt count} tells how many subsequences from each of lists should be
taken i.e. how many times should be operation of taking {\tt len1} and {\tt len2}
nodes repeated. If both lists are empty function returns an empty list.
\vspace{\baselineskip}

\subsubsection*{Example}

\begin{lstlisting}[language=C++]
Input:
//info, keys for every node are the same
s1 = {1, 2, 3, 4, 5, 6, 7, 8, 9}
s2 = {10, 20, 30, 40, 50, 60, 70, 80, 90, 100, 110}

start1 = 2
start2 = 1

len1 = 3
len2 = 2

count = 4

Output:
//only info as keys are the same
Merge(s1, 2, 3, s2, 1, 2, 4) =
{3, 4, 5, 20, 30, 6, 7, 8, 40, 50, 9, 60, 70, 80, 90}
\end{lstlisting}
\newpage

\section{Decisions and changes}

On the early stage my implementation had {\tt operator[]} overloading. The idea was
to make accessing nodes more intuitive and to make implementation of other methods
easier. However this approach is not suitable for structure like linked list.
As list is not a random access structure it is necessary to iterate through nodes
of list to obtain specified node. Using this operator for example in removing all
occurances of nodes with chosen key could be an operation with very high time complexity.
The second problem is breaking encapsulation. With such a public operator node's {\tt next}
pointer could be changed in the outside of {\tt Sequence} and it could lead to list's
breaking (and in result of memory leak and losing access to some nodes).

\indent
With abandonment of {\tt operator[]} new problem arised - accessing nodes of lists
in {\tt Merge} function. As it is impossible due to {\tt Node} being private I had
to implement a method which returns a fragment of list. Then with use of overloaded
{\tt +=} operator it was possible to easily make a new list which is a combination of fragments
of two other lists.

\indent
To make testing possible I also implemented {\tt Print} method which prints all values
stored in the list. There is one restriction - data types stored in the list has to be
valuable (or with {\tt <<} operator overloaded). I considered an external method for this
purpose but without access to the elements of list it is impossible.

\section{Testing}

\subsection{Introduction}

I decided to use Catch2 testing framework to get familiar with using such frameworks.
With its help I divided tests on several test cases. These cases are also divided into
sections. That makes them easier to be performed as in case of failure framework
gives us exact information about test case and section in which assertion has been failed.

\indent
Unit test are performed in two source files, one for {\tt Sequence} methods testing and the second
for {\tt Merge} function testing.

\indent
For the purpose of memory leaks checking I used Valgrind software.

\subsection{Structure of tests}

\subsubsection*{Empty list}

\begin{itemize}
    \item Printing list
    \item Checking if list is empty
    \item Using removal methods
    \item Using {\tt Find} method
\end{itemize}

\subsubsection*{Filling list}

\begin{itemize}
    \item Adding nodes on the beginning
    \item Adding nodes on the end
    \item Adding nodes inside the list
\end{itemize}

\subsubsection*{Removing nodes from list}

\begin{itemize}
    \item Removing nodes from the beginning
    \item Removing nodes from the end
    \item Removing selected nodes
    \item Removing all nodes
    \item Removing all nodes with chosen key
\end{itemize}

\subsubsection*{Comparison operators}

\begin{itemize}
    \item Equality operator
    \item Inequality operator
\end{itemize}

\subsubsection*{Copying lists}

\begin{itemize}
    \item Copying with copy constructor
    \item Copying with assignment operator
\end{itemize}

\subsubsection*{Other methods}

\begin{itemize}
    \item Size method
    \item Find method
    \item Checking number of occurances
\end{itemize}

\subsubsection*{Merge function}

\begin{itemize}
    \item Merging when both of lists are empty
    \item Merging when one of lists is empty
    \item Merging with filled lists
\end{itemize}

\end{document}