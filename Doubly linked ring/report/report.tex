\documentclass{article}

\usepackage{listings}
\usepackage{xcolor}
\usepackage{geometry}

\title{Doubly linked circular list - an EADS project}
\author{Maciej Marcinkiewicz}
\date{January 2020}

\newgeometry{lmargin=3.2cm, rmargin=3.2cm, bmargin=2.5cm}

\lstset{
    tabsize=4, % tab space width
    showstringspaces=false, % don't mark spaces in strings
    % numbers=left, % display line numbers on the left
    basicstyle=\ttfamily,
    keywordstyle=\color{blue}\ttfamily,
    stringstyle=\color{red}\ttfamily,
    commentstyle=\color{green}\ttfamily,
    % morecomment=[l][\color{magenta}]{\#}
}

\begin{document}

\maketitle

\section{Introduction}

The goal of the project is to implement doubly linked circular list (or as some call
it – ring) in the C++ programming language with use of templates. As in the previous 
project the requirement is to create a container with two generic parameters – {\tt Key}
and {\tt Info}. One more requirement was to use iterators, so {\tt Ring} has to conatin
{\tt iterator} and {\tt const\_iterator} classes.

\indent
The second task is to create {\tt Split function} which produces two rings out of one.
This function is meant to be external (and cannot be linked with keyword {\tt friend}).

\section{Overview of funtions}

\subsection{Constructors and destructor}

\begin{lstlisting}[language=C++]
Ring()
\end{lstlisting}
Default constructor.

\begin{lstlisting}[language=C++]
~Ring()
\end{lstlisting}
Destructor.

\begin{lstlisting}[language=C++]
Ring(const Ring<Key, Info> &other)
\end{lstlisting}
Copy constructor.

\begin{lstlisting}[language=C++]
Data(const Key &key, const Info &info)
\end{lstlisting}
Constructor for {\tt Data}. Creates new {\tt Data} object with given key and information.

\begin{lstlisting}[language=C++]
Node(const Key &key, const Info &info, Node *next, Node *previous)
     : data(key, info)
\end{lstlisting}
Constructor for {\tt Node}. Assigns node's pointers.

\begin{lstlisting}[language=C++]
iterator()
const_iterator()
\end{lstlisting}
Constructors for iterator classes.
\vspace{\baselineskip}

\subsection{Overloaded operators}

\begin{lstlisting}[language=C++]
Sequence<Key, Info> &operator=(const Sequence<Key, Info> &other)
\end{lstlisting}
Assignment operator.

\newpage
\begin{lstlisting}[language=C++]
bool operator==(const Ring<Key, Info> &other)
\end{lstlisting}
Comparison operator. Returns {\tt true} if compared rings contain the same elements.

\begin{lstlisting}[language=C++]
bool operator!=(const Ring<Key, Info> &other)
\end{lstlisting}
Comparison operator. Returns {\tt false} if compared rings contain the same elements.

\begin{lstlisting}[language=C++]
bool iterator::operator==(const Ring<Key, Info> &other)
bool const_iterator::operator==(const Ring<Key, Info> &other)
\end{lstlisting}
Comparison operators. Return {\tt true} if compared nodes of iterators are the same.

\begin{lstlisting}[language=C++]
bool iterator::operator!=(const Ring<Key, Info> &other)
bool const_iterator::operator!=(const Ring<Key, Info> &other)
\end{lstlisting}
Comparison operators. Return {\tt true} if compared nodes of iterators are the same.

\begin{lstlisting}[language=C++]
Data &iterator::operator*() const
Data &const_iterator::operator*() const
Data *iterator::operator->() const
Data *const_iterator::operator->() const
\end{lstlisting}
Dereference operators for iterators.

\begin{lstlisting}[language=C++]
iterator &iterator::operator++()
const_iterator &const_iterator::operator++()
iterator iterator::operator++(int)
const_iterator const_iterator::operator++(int)
\end{lstlisting}
Preincrementation and postincrementation operators for iterators.

\begin{lstlisting}[language=C++]
iterator &iterator::operator--()
const_iterator &const_iterator::operator--()
iterator iterator::operator--(int)
const_iterator const_iterator::operator--(int)
\end{lstlisting}
Predecrementation and postdecrementation operators for iterators.

\subsection{Adding elements}

\begin{lstlisting}[language=C++]
void PushBack(const Key &key, const Info &info)
\end{lstlisting}
Method which adds new node of given key and information behind {\tt first}.

\begin{lstlisting}[language=C++]
bool Insert(const Key &key, const Info &info, 
            const iterator &position)
\end{lstlisting}
Method which adds new node of given key and information before {\tt position}. Returns
{\tt false} if {\tt position} is {\tt end()}.
\vspace{\baselineskip}

\subsection{Removing elements}

\begin{lstlisting}[language=C++]
bool PopBack()
\end{lstlisting}
Method which removes node behind {\tt first}. Returns {\tt false} if list is empty.

\begin{lstlisting}[language=C++]
bool Remove(const iterator &position)
\end{lstlisting}
Method which removes node which is pointed by {\tt position}. Returns {\tt false} if list is empty
or if {\tt position} is {\tt end()}.

\begin{lstlisting}[language=C++]
bool Clear()
\end{lstlisting}
Removes all nodes. Returns {\tt false} if list is empty.

\begin{lstlisting}[language=C++]
bool RemoveAllOccurances(const Key &key)
\end{lstlisting}
Method which removes all nodes with {\tt key}. Returns {\tt false} if list is empty
or there are no nodes with such a key.
\vspace{\baselineskip}

\subsection{Other methods}

\begin{lstlisting}[language=C++]
iterator Find(const Key &key, int whichOccurance = 1)
const_iterator Find(const Key &key, int whichOccurance = 1)
\end{lstlisting}
Returns iterator pointing on element with {\tt key}. {\tt whichOccurance}
tells which occurance of key has to be found. If there is no such a node returns {\tt end()}.

\begin{lstlisting}[language=C++]
bool IsEmpty() const
\end{lstlisting}
Returns {\tt true} if list contains no nodes.

\begin{lstlisting}[language=C++]
int Size() const
\end{lstlisting}
Returns private field {\tt size} which conatins number of nodes in list.

\begin{lstlisting}[language=C++]
int NumberOfOccurances(const Key &key) const
\end{lstlisting}
Returns number of occurances of nodes with given key.
\vspace{\baselineskip}

\subsection{Iterator methods}

\begin{lstlisting}[language=C++]
iterator begin()
const_iterator begin() const
\end{lstlisting}
Return iterator pointing on {\tt first}.

\begin{lstlisting}[language=C++]
iterator end()
const_iterator end() const
\end{lstlisting}
Return iterator pointing on {\tt nullptr}.
\vspace{\baselineskip}

\subsection{Split function}

\begin{lstlisting}[language=C++]
void Split(const Ring<Key, Info> &source, bool direction,
           Ring<Key, Info> &result1, int sequence1, int rep1,
           Ring<Key, Info> &result2, int sequence2, int rep2)
\end{lstlisting}
Produces two rings from another one. {\tt source} is ring from other are produced, 
{\tt direction} tells if passage should be forward ({\tt true}) or backward ({\tt false}),
{\tt result1} and {\tt result2} are rings in which produced ones are stored,
{\tt sequence1}and {\tt sequence2} tell how many nodes are taken in which passage
and eventually {\tt rep1} and {\tt rep2} tell how many passages have to be taken.
If result1 or result2 are not empty, they are cleared.
\newpage

\subsubsection*{Example}

\begin{lstlisting}[language=C++]
Input:
//info, keys for every node are the same
source = {1, 2, 3, 4, 5, 6, 7, 8, 9, 10, 11, 12}

direction = false

result1 = r1
result2 = r2

sequence1 = 3
sequence2 = 2

rep1 = 2
rep2 = 4

Output:
//only info as keys are the same
r1 = {1, 12, 11, 8, 7, 6}
r2 = {10, 9, 5, 4, 3, 2, 1, 12}
\end{lstlisting}

\end{document}